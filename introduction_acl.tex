\RequirePackage[]{lineno}
\documentclass[12pt]{article}
\usepackage{caption}
\usepackage{times}
\usepackage{setspace}
\usepackage{longtable}
\usepackage{amsmath}
\usepackage{booktabs}
\usepackage{float}
\usepackage{mathpazo}
\usepackage{times}
\usepackage{tikz}
\usepackage{graphicx}
\usepackage[hmargin=2.25cm, vmargin=2cm, headheight=15.5pt]{geometry}
\usepackage{multirow}
\usepackage{tcolorbox}
\usepackage{multicol}
\usepackage{tabularx}
\usepackage{rotating}
\usepackage{pdflscape}


\captionsetup[figure]{font=small}
\captionsetup[table]{font=small}

\usetikzlibrary{arrows,calc}
\geometry{margin=1in}

%\captionsetup{font=doublespacing, size= footnotesize}% Double-spaced float captions
\doublespacing
\DeclareCaptionJustification{double}{\DoubleSpacing}
% Reasonable page setup


\usepackage[]{natbib}
\bibpunct[; ]{(}{)}{;}{a}{,}{;}

% to avoid things being lost to overleaf comment bubbles
\long\def\authornote#1{%
    \leavevmode\unskip\raisebox{-3.5pt}{\rlap{$\scriptstyle\dagger$}}%
    \marginpar{\raggedright\hbadness=10000
        \def\baselinestretch{0.8}\tiny
        \it #1\par}}
\newcommand{\DBS}[1]{\authornote{DBS: #1}}
\newcommand{\ACL}[1]{\authornote{ACL: #1}}

\usepackage{authblk}
\renewcommand\Affilfont{\small}

\newenvironment{abox}[1]{
  \begin{tcolorbox}[float,title=#1, colback=blue!4]
  \fontsize{9}{10}\selectfont
  \begin{multicols}{2}
}{
  \end{multicols}
  \end{tcolorbox}
}


\newenvironment{ecolettcover}{\maketitle}{\clearpage}
\newenvironment{ecolettabstract}{\clearpage\section*{Abstract}}{\clearpage}
\tikzset{
	%Define standard arrow tip
	>=stealth',
	%Define style for different line styles
	help lines/.style={dashed, thick},
	axis/.style={<->},
	important line/.style={thick},
	connection/.style={thick, dotted},
}

%\title{The structural sensitivity of competition models: how model formulation changes our predictions of species coexistence}
\title{General Introduction}
\author[1]{Alba Cervantes-Loreto}


% Include the date command, but leave its argument blank.
\date{}

%%%%%%%%%%%%%%%%% END OF PREAMBLE %%%%%%%%%%%%%%%%
\let\oldequation\equation
\let\oldendequation\endequation

\renewenvironment{equation}
  {\linenomathNonumbers\oldequation}
  {\oldendequation\endlinenomath}

% \pagestyle{empty}

\begin{document}
\linenumbers
% Double-space the manuscript.
\baselineskip30pt
\maketitle

\section*{Models of biotic interactions}
Interactions between organisms underpin the persistence of almost all life forms on Earth \citep{lawton1999there}. Furthermore, a large body of work has shown that biotic interactions determine emergent properties of natural systems, such as stability \citep{may1972will, wootton2016many,song2018will}, resilience \citep{capdevila2021reconciling}, ecosystem functioning \citep{turnbull2013coexistence,godoy2020excess}, and the coexistence of multiple species \citep{chesson2000mechanisms,saavedra2017structural}. Unsurprisingly, numerous ecological and evolutionary concepts revolve around the effects that organisms exert on each other \citep{gause_experimental_1934,macarthur1967limiting,thompson1999evolution, hillerislambers2012rethinking, chase2009ecological}.
%concepts: competitive exclusion, limiting similarity, community assembly, ecological niche, co-evolution

Describing the effect of biotic interactions often requires the use of mathematical models to represent them \citep{maynard1978models,rossberg2019let}. Mathematical descriptions of interactions are ``useful fictions'' \citep{box2011statistical} in a twofold manner. First, they create a description of how organisms that coincide in space and time affect each other. Almost all known types of interactions can be described in the form of mathematical expressions that reproduce the observed data faithfully \citep{volterra1926fluctuations,holling1959some,holt1977predation,adler2018competition,wood1999super,holland2002population,vazquez2005interaction,stouffer2021hidden} . For example, the effect neighboring plants have on each other can be accurately described in various natural systems with individual fitness models that include solely the densities of the interacting species embedded in a function of negative density dependence \citep{adler2018competition,hart2018quantify}. Second, models are practical tools with which to make predictions beyond the phenomena they describe and thus, provide general insights into how natural systems operate \citep{evans2012predictive,stouffer2019all}. For instance, models that describe competitive interactions between plants have been extensively used to demonstrate the mechanisms that maintain diversity when species compete for the same pool of resources \citep{levine2009importance,godoy_phylogenetic_2014, godoy_phenology_2014, stouffer2018cyclic,bimler_accurate_2018}

\section*{Are simple models always better?}
Models that capture the effect of biotic interactions are abstractions of reality, and abstractions always reflect choices \citep{levins2006strategies}. Building models that include all aspects of reality is not only impractical but also unfeasible.  Therefore, ecologists and evolutionary biologists have to continuously make choices regarding which variables to include in a model and which to omit \citep{evans2012predictive, rossberg2019let}. A common assumption when building models is that to achieve general insights, we should favor simple models \citep{evans2013simple}. Indeed there is a general belief in ecology and evolution that a good model should include as little as possible \citep{evans2013simple,orzack2012philosophy}. This belief is often rooted in an implicit philosophical stance that one can not maximize generality (i.e., models that apply to more than one system) and realism (i.e., models that produce accurate predictions for a system) \citep{levins2006strategies,evans2012predictive}.


Inevitably, model building in biology leads to a key question that will, in turn, modify the outcomes achieved by any model: when is a model ``realistic'' enough \citep{stouffer2019all}? The answer to this question will depend on the purpose for which a model is built. Models that capture the effect of biotic interactions tend to fall into the category of ``demonstration models''. These types of models are often based on phenomenological descriptions of processes and have the general aim to show that the modeled principles are sufficient to produce the phenomena of interest \citep{evans2013simple}.  Demonstration models however, do not help decide whether the modelled principles are \textit{necessary} \citep{evans2013simple}. The task to decide the necessary principles and thus the answer to the question of when a model is realistic enough becomes the modeler's responsibility. In many cases, the answer to this question can appear arbitrary or solely determined by the predominant paradigm regarding the studied system \citep{holland2006comment,bascompte2006response,kokko2007ecogenetic,aladwani2019addition,mayfield2017higher,martyn2021identifying}.



Always favoring simple models in ecological and evolutionary studies can be problematic from two perspectives. First, the assumption that more complex models do not lead to general insights is seldomly tested. For example, most models that capture competitive interactions between plants have the implicit assumption that competitive effects between individuals are always additive and direct \citep{schoener1974some,freckleton2001predicting,kraft2015plant}.  However, when models and data collection were set up to capture non-additive effects of interactions between individuals of co-occurring species, the evidence overwhelmingly showed that including these levels of biotic complexity was necessary to capture plant interactions accurately \citep{mayfield2017higher,martyn2021identifying,lai2021non}.  Thus, in some cases, increasing complexity increases rather than hampers the general insights obtained from models of biotic interactions.

Second, failing to include necessary levels of complexity can hinder our ability to predict how natural communities will react to novel conditions. Predictions of how natural systems will behave in the future are inherently challenging \citep{sutherland2006predicting}. Nevertheless, ignoring heterogeneities at various levels can further complicate rather than simplify predictions  \citep{evans2012predictive}. For instance, demographic models tend to treat ecological and evolutionary dynamics separately, despite the general understanding that both processes are often intertwined \citep{macarthur1962some,kokko2007ecogenetic}. Ignoring eco-evolutionary feedbacks leads to predictions that are inconsistent with empirical data and produce counterintuitive results in novel conditions \citep{kokko2007ecogenetic}. Thus, the implicit assumption that good models should include as little as possible should be treated with caution in ecological and evolutionary contexts \citep{evans2013simple, kokko2007ecogenetic,abrams2001describing}.


\section*{Thesis aims}

Despite arguments in favor of increasing realism in models of biotic interactions, doing so remains a challenge in many ecological and evolutionary studies.  Part of these difficulties arises from the lack of appropriate theoretical frameworks that allow the inclusion of different levels of complexity \citep{mayfield2017higher,martyn2021identifying}. Thus, it remains unclear whether increasing model realism is warranted for many natural systems. In this thesis, I propose theoretical, statistical, and empirical frameworks to increase realism in models of biotic interactions to understand when higher levels of complexity are justified. Furthermore, I also explore the consequences of increasing model realism in predictions related to diversity maintenance at ecological and evolutionary scales. The individual chapters of this thesis are thematically broad, but all address in a different way the challenges and consequences of incorporating biotic, abiotic, and mathematical complexity in models of biotic interactions.

\section*{Including biotic, abiotic and mathematical complexity}
Perhaps one of the most challenging aspects of building more realistic models of biotic interactions is the lack of theoretical frameworks that allow incorporating complexities observed in empirical contexts \citep{abrams1983arguments}.  Such is the case of competition between pollinators that forage for the same resources \citep{thomson_importance_2020}. Despite an overwhelmingly amount of empirical evidence showing that pollinators modify their foraging behavior in the presence of other foraging species \citep{morse_resource_1977,inouye_resource_1978,thompson_dynamics_2006,brosi_single_2013,briggs_competitive_2016}, incorporating these behavioral changes into population dynamics remains elusive \citep{thomson_importance_2020}. 
%\section*{Concluding remarks}
%The individual chapters of this thesis are thematically broad but all address in a different way the consequences of increasing complexity in models of biotic interactions. With the exception of  I explore the consequences in terms of the coexistence of organisms. Through out this thesis I explored different ecological systems, with different types of interactions and species in them. However, the fundamental questions remains : what happens when add biological, environmental and mathematical complexity to the study of species interactions? Do they change our predictions?

% As scientists, narrative reasoning allows us to explore, at a high level, the possible trajectories that evolution may take.


\clearpage
\bibliographystyle{ecology_letters}
\bibliography{Bibliography.bib}

\end{document}
