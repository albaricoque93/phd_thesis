\RequirePackage[]{lineno}
\documentclass[12pt]{article}
\usepackage{caption}
\usepackage{times}
\usepackage{setspace}
\usepackage{longtable}
\usepackage{amsmath}
\usepackage{booktabs}
\usepackage{float}
\usepackage{mathpazo}
\usepackage{times}
\usepackage{tikz}
\usepackage{graphicx}
\usepackage[hmargin=2.25cm, vmargin=2cm, headheight=15.5pt]{geometry}
\usepackage{multirow}
\usepackage{tcolorbox}
\usepackage{multicol}
\usepackage{tabularx}
\usepackage{rotating}
\usepackage{pdflscape}


\captionsetup[figure]{font=small}
\captionsetup[table]{font=small}

\usetikzlibrary{arrows,calc}
\geometry{margin=1in}

%\captionsetup{font=doublespacing, size= footnotesize}% Double-spaced float captions
\doublespacing
\DeclareCaptionJustification{double}{\DoubleSpacing}
% Reasonable page setup


\usepackage[]{natbib}
\bibpunct[; ]{(}{)}{;}{a}{,}{;}

% to avoid things being lost to overleaf comment bubbles
\long\def\authornote#1{%
    \leavevmode\unskip\raisebox{-3.5pt}{\rlap{$\scriptstyle\dagger$}}%
    \marginpar{\raggedright\hbadness=10000
        \def\baselinestretch{0.8}\tiny
        \it #1\par}}
\newcommand{\DBS}[1]{\authornote{DBS: #1}}
\newcommand{\ACL}[1]{\authornote{ACL: #1}}

\usepackage{authblk}
\renewcommand\Affilfont{\small}

\newenvironment{abox}[1]{
  \begin{tcolorbox}[float,title=#1, colback=blue!4]
  \fontsize{9}{10}\selectfont
  \begin{multicols}{2}
}{
  \end{multicols}
  \end{tcolorbox}
}


\newenvironment{ecolettcover}{\maketitle}{\clearpage}
\newenvironment{ecolettabstract}{\clearpage\section*{Abstract}}{\clearpage}
\tikzset{
	%Define standard arrow tip
	>=stealth',
	%Define style for different line styles
	help lines/.style={dashed, thick},
	axis/.style={<->},
	important line/.style={thick},
	connection/.style={thick, dotted},
}

%\title{The structural sensitivity of competition models: how model formulation changes our predictions of species coexistence}
\title{Introduction}
\author[1]{Alba Cervantes-Loreto}


% Include the date command, but leave its argument blank.
\date{}

%%%%%%%%%%%%%%%%% END OF PREAMBLE %%%%%%%%%%%%%%%%
\let\oldequation\equation
\let\oldendequation\endequation

\renewenvironment{equation}
  {\linenomathNonumbers\oldequation}
  {\oldendequation\endlinenomath}

% \pagestyle{empty}

\begin{document}
\linenumbers
% Double-space the manuscript.
\baselineskip30pt
\maketitle


Interactions between organisms underpin the persistence of almost all life forms on Earth \citep{lawton1999there}. Furthermore, a large body of work has shown that biotic interactions determine emergent properties of natural systems, such as stability \citep{may1972will, wootton2016many,song2018will}, resilience \citep{capdevila2021reconciling}, ecosystem functioning \citep{turnbull2013coexistence,godoy2020excess}, and the coexistence of multiple species \citep{chesson2000mechanisms,saavedra2017structural}. Unsurprisingly, numerous ecological and evolutionary concepts revolve around the reciprocal forces that organisms exert on each other \citep{gause_experimental_1934,macarthur1967limiting,thompson1999evolution, hillerislambers2012rethinking, chase2009ecological}.
%concepts: competitive exclusion, limiting similarity, community assembly, ecological niche, co-evolution

Capturing the effect of biotic interactions often requires the use of mathematical models to represent them \citep{maynard1978models}. Mathematical descriptions of interactions are ``useful fictions'' \citep{box2011statistical} in a twofold manner. First, they create a description of how organisms that coincide in space and time reciprocally affect each other. Almost all known types of interactions can be described in the form of mathematical expressions that reproduce the observed data faithfully \citep{volterra1926fluctuations,holling1959some,holt1977predation,adler2018competition,wood1999super,holland2002population,vazquez2005interaction,stouffer2021hidden} . For example, the effect neighboring plants have on each other can be accurately described in various natural systems with individual fitness models that include solely the densities of the interacting species embedded in a function of negative density dependence \citep{adler2018competition,hart2018quantify}. Second, models are practical tools with which to make predictions beyond the phenomena they describe and thus, provide general insights into how natural systems operate \citep{evans2012predictive,stouffer2019all}. For instance, models that describe competitive interactions between plants have been extensively used to demonstrate the mechanisms that maintain diversity when species compete for the same pool of resources \citep{levine2009importance,godoy_phylogenetic_2014, godoy_phenology_2014, stouffer2018cyclic,bimler_accurate_2018}.


Models that capture the effect of biotic interactions are abstractions of reality and abstractions always reflect choices \citep{levins2006strategies}. Including all aspects of reality in a model is not only impractical but also unfeasible, therefore ecologists and evolutionary biologists have to continuously make choices regarding which variables to include in a model and which to omit \citep{evans2012predictive}. A common assumption when building models is that to achieve general insights, we should favor simple models \citep{evans2013simple}. Indeed there is a general belief in ecology and evolution that a good model should include as little as possible \citep{evans2013simple,orzack2012philosophy}. This belief is often rooted in an implicit philosophical stance that one can not maximize generality (i.e., models that apply to more than one system) and realism (i.e., models that produce accurate predictions for a system) \citep{levins2006strategies,evans2012predictive}.


Inevitably, model building in biology leads to a key question that will in turn modify the outcomes achieved by any model: when is a model ``realistic'' enough \citep{stouffer2019all}? The answer to this question will depend on the purpose a model is built for. Models that capture the effect of biotic interactions often fall into the category of ``demonstration models''. These types of models are often based on phenomenological descriptions of processes and have the general aim to show that the modeled principles are sufficient to produce the phenomena of interest \citep{evans2013simple}.  Demonstration models however, do not help decide whether the modelled principles are \textit{necessary} \citep{evans2013simple}. The task to decide what are the necessary principles and thus the answer to the question of when is a model realistic enough becomes the responsibility of the modeler. In many cases, the answer to this question can appear arbitrary or solely determined by the predominant paradigm regarding the studied system \citep{holland2006comment,bascompte2006response,aladwani2019addition,martyn2021identifying}.



Always favoring simple models in ecological and evolutionary studies can be problematic from two perspectives. First, the assumption that more complex models do not lead to general insights is seldomly tested. For example, most models that capture competitive interactions between plants have the implicit assumption that competitive effects between individuals are always additive and direct \citep{schoener1974some,freckleton2001predicting,kraft2015plant}. Consequently, the vast majority of experimental designs for quantifying plant interactions for individual fitness models evaluate how a focal species’ fitness changes with a single competitor species  \citep{freckleton2001predicting,hart2018quantify}. However, when models and data collection were set up to capture non-additive, per capita effects of interactions between individuals of co-occurring species and not only direct pairwise interactions, evidence overwhelmingly showed that including these levels of biotic complexity was necessary to accurately capture plant interactions \citep{mayfield2017higher,martyn2021identifying,lai2021non}. Therefore, models that only include direct pairwise interactions between individuals create an over-simplified and potentially inaccurate depiction of plant interactions, and in turn, of our understanding of how competitive dynamics shape communities \citep{martyn2021identifying}. From this perspective, increasing complexity increases rather than hampers the general insights obtained from models of biotic interactions.


Second, increasing realism in models of biotic interactions could dramatically change predictions made from these models. Predictions of how natural systems will behave in the future is inherently challenging \citep{sutherland2006predicting,evans2012predictive}. Nevertheless, failing to include necessary levels of complexity can further hinder our ability to understand how natural communities will react in novel conditions. For example, the level of complexity used to describe interactions between mutualistic partners will fundamentally determine the number of species that can coexist  and thus if a new species can enter a community  \citep{holland2006comment,bascompte2006response, bastolla2009architecture}.

This thesis is focused on exploring with theoretical and empirical tools 



\section*{Concluding remarks}
The individual chapters of this thesis are thematically broad but all address in a different way the consequences of increasing complexity in models of biotic interactions. With the exception of  I explore the consequences in terms of the coexistence of organisms. Through out this thesis I explored different ecological systems, with different types of interactions and species in them. However, the fundamental questions remains : what happens when add biological, environmental and mathematical complexity to the study of species interactions? Do they change our predictions?

% As scientists, narrative reasoning allows us to explore, at a high level, the possible trajectories that evolution may take.



\clearpage
\bibliographystyle{ecology_letters}
\bibliography{Bibliography.bib}

\end{document}
