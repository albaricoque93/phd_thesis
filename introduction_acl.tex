\RequirePackage[]{lineno}
\documentclass[12pt]{article}
\usepackage{caption}
\usepackage{times}
\usepackage{setspace}
\usepackage{longtable}
\usepackage{amsmath}
\usepackage{booktabs}
\usepackage{float}
\usepackage{mathpazo}
\usepackage{times}
\usepackage{tikz}
\usepackage{graphicx}
\usepackage[hmargin=2.25cm, vmargin=2cm, headheight=15.5pt]{geometry}
\usepackage{multirow}
\usepackage{tcolorbox}
\usepackage{multicol}
\usepackage{tabularx}
\usepackage{rotating}
\usepackage{pdflscape}


\captionsetup[figure]{font=small}
\captionsetup[table]{font=small}

\usetikzlibrary{arrows,calc}
\geometry{margin=1in}

%\captionsetup{font=doublespacing, size= footnotesize}% Double-spaced float captions
\doublespacing
\DeclareCaptionJustification{double}{\DoubleSpacing}
% Reasonable page setup


\usepackage[]{natbib}
\bibpunct[; ]{(}{)}{;}{a}{,}{;}

% to avoid things being lost to overleaf comment bubbles
\long\def\authornote#1{%
    \leavevmode\unskip\raisebox{-3.5pt}{\rlap{$\scriptstyle\dagger$}}%
    \marginpar{\raggedright\hbadness=10000
        \def\baselinestretch{0.8}\tiny
        \it #1\par}}
\newcommand{\DBS}[1]{\authornote{DBS: #1}}
\newcommand{\ACL}[1]{\authornote{ACL: #1}}

\usepackage{authblk}
\renewcommand\Affilfont{\small}

\newenvironment{abox}[1]{
  \begin{tcolorbox}[float,title=#1, colback=blue!4]
  \fontsize{9}{10}\selectfont
  \begin{multicols}{2}
}{
  \end{multicols}
  \end{tcolorbox}
}


\newenvironment{ecolettcover}{\maketitle}{\clearpage}
\newenvironment{ecolettabstract}{\clearpage\section*{Abstract}}{\clearpage}
\tikzset{
	%Define standard arrow tip
	>=stealth',
	%Define style for different line styles
	help lines/.style={dashed, thick},
	axis/.style={<->},
	important line/.style={thick},
	connection/.style={thick, dotted},
}

%\title{The structural sensitivity of competition models: how model formulation changes our predictions of species coexistence}
\title{Increasing realism in models of biotic interactions: ecological and evolutionary consequences}
\author[1]{Alba Cervantes-Loreto}


% Include the date command, but leave its argument blank.
\date{}

%%%%%%%%%%%%%%%%% END OF PREAMBLE %%%%%%%%%%%%%%%%
\let\oldequation\equation
\let\oldendequation\endequation

\renewenvironment{equation}
  {\linenomathNonumbers\oldequation}
  {\oldendequation\endlinenomath}

% \pagestyle{empty}

\begin{document}
\linenumbers
% Double-space the manuscript.
\baselineskip30pt
\maketitle

\section*{Abstract}
Interactions between organisms give rise to emergent properties of natural systems. This underpins the ubiquity of biotic interactions in the study of ecological and evolutionary dynamics. The representation of biotic interactions often requires models and simplifying assumptions since it is impossible to account for all aspects of the world in a single model. Critical choices, such as the number of species that can alter the interaction between a focal pair or which abiotic variables constitute the environment, are necessary when building ecological and evolutionary models. Such simplifying assumptions inevitably lead to the omission of heterogeneities at various levels. Complexity that is unaccounted for can, in turn, make the relationships between organisms appear noisy and fundamentally change model-based predictions. Despite this, ecological and evolutionary studies often lack appropriate frameworks that allow the inclusion of different levels of complexity in representations of biotic interactions. Thus, it is unclear whether including more realistic assumptions is warranted for the vast majority of natural systems. In this thesis, I explore how incorporating complexity as abiotic and biotic modifiers, as well as different sources of uncertainty, reveals potential explanatory generalities in natural systems. I also explore how accounting for these variables changes predictions related to the maintenance of diversity at ecological and evolutionary scales. Throughout this thesis, I focus on different types of interactions and organisms and propose mathematical and statistical frameworks that can be used beyond the studied systems.


In \textbf{Chapter 2}, I explore how the presence of multiple species and different environmental contexts change the strength of plant-pollinator interactions. I propose a framework for using pollinator functional responses to examine the role of pollinator-pollinator interactions and abiotic conditions in altering the time between floral visits of a focal pollinator. I show that both density dependent responses and abiotic conditions are necessary to explain the number of visits a pollinator makes. In \textbf{Chapter 3},  I explore how incorporating different sources of uncertainty changes predictions of species coexistence. I do this by simultaneously exploring how different model formulations, environmental contexts, and parameter uncertainty change the probability of predicting coexistence in an experimental system. I provide direct evidence that predictions of species coexistence are likely to change given the models used to quantify density-dependence. I also provide a theoretical framework to explore predictions made with different models. Finally, in \textbf{Chapter 4}, I adopt an ecological framework to examine the evolutionary dynamics of sexually antagonistic alleles. I show that incorporating environmental fluctuations can substantially increase the amount of genetic diversity in a population under sexually antagonistic selection. Overall, the results of my thesis show that the assumptions adopted by some ecological and evolutionary models tend to be oversimplifying. Here, I provide tools for ecologists and evolutionary biologists to explore  more realistic representations of biotic interactions as well as their consequences for diversity maintenance.

\clearpage
\section*{Preface}

My thesis has been prepared as a collection of three standalone scientific articles. Each chapter is a standalone piece of research and, therefore, I only provide a general Introduction and Conclusion chapters linking the three chapters together. In  \textbf{Chapter 1}, I focus on describing how my three chapters are connected. In \textbf{Chapter 5}, I focus on summarising the results from each of my thesis chapters and their combined implications both in both how we study interactions and their consequences for diversity maintenance. Finally, I further expand on new ideas beyond those presented in the different chapters to discuss about the future steps moving forward.


At the time of thesis submission, each of these three articles are in different stages of the publication process.

\textbf{Chapter 2}: ``The context dependency of pollinator interference: how environmental conditions and co-foraging species impact floral visitaion'' was published in May 2021 in the journal \textit{Ecology Letters} in volume 24, no. 7, pages 1443--1454.

\textbf{Chapter 3}: ``The interplay of environmental conditions, parameter sensitivity and structural sensitivity in predictions of species coexistence'' is in preparation for submission to \textit{Ecology Letters}.

\textbf{Chapter 4}: ``Quantifying the relative contributions of environmental fluctuations to the maintenance of a sexually antagonistic polymorphism'' is in preparation for submission to \textit{The American Naturalist}.

\clearpage
\section*{Chapter 1: General Introduction}

\subsection*{Models of biotic interactions}
Interactions between organisms underpin the persistence of almost all life forms on Earth \citep{lawton1999there}. Furthermore, a large body of work has shown that biotic interactions determine emergent properties of natural systems, such as stability \citep{may1972will, wootton2016many,song2018will}, resilience \citep{capdevila2021reconciling}, ecosystem functioning \citep{turnbull2013coexistence,godoy2020excess}, and the coexistence of multiple species \citep{chesson2000mechanisms,saavedra2017structural}. Unsurprisingly, numerous ecological and evolutionary concepts revolve around the effects that organisms exert on each other \citep{gause_experimental_1934,macarthur1967limiting,thompson1999evolution, hillerislambers2012rethinking, chase2009ecological}.
%concepts: competitive exclusion, limiting similarity, community assembly, ecological niche, co-evolution

From their origin as natural sciences, the disciplines of ecology and evolution have shifted from a descriptive towards a more predictive and quantitative approach. This shift brought with it the use of mathematical models to describe natural phenomena \citep{maynard1978models,rossberg2019let}. Mathematical descriptions of interactions are ``useful fictions'' \citep{box2011statistical} in a twofold manner. First, they create a description of how organisms that coincide in space and time affect each other. Almost all known types of interactions can be described in the form of mathematical expressions that reproduce the observed data faithfully \citep{volterra1926fluctuations,holling1959some,holt1977predation,adler2018competition,wood1999super,holland2002population,vazquez2005interaction,stouffer2021hidden}  Second, models are practical tools with which to make predictions beyond the phenomena they describe and thus, provide general insights into how natural systems operate \citep{evans2012predictive,stouffer2019all,rossberg2019let}.

%For example, the effect neighboring plants have on each other can be accurately described in various natural systems with individual fitness models that include solely the densities of the interacting species embedded in a function of negative density dependence \citep{adler2018competition,hart2018quantify}.

%For instance, models that describe competitive interactions between plants have been extensively used to demonstrate the mechanisms that maintain diversity when species compete for the same pool of resources \citep{levine2009importance,godoy_phylogenetic_2014, godoy_phenology_2014, stouffer2018cyclic,bimler_accurate_2018}

\subsection*{The perils of simple models}
Models that capture the effect of biotic interactions are abstractions of reality, and abstractions always reflect choices \citep{levins2006strategies}. Building models that include all aspects of reality is not only impractical but also unfeasible.  Therefore, ecologists and evolutionary biologists have to continuously make choices regarding which variables to include in a model and which to omit \citep{evans2012predictive, rossberg2019let}. A common assumption when building models is that to achieve general insights, we should favor simple models \citep{evans2013simple}. Indeed there is a general belief in ecology and evolution that a good model should include as little as possible \citep{evans2013simple,orzack2012philosophy}. This belief is often rooted in an implicit philosophical stance that one can not maximize generality (i.e., models that apply to more than one system) and realism (i.e., models that produce accurate predictions for a system) \citep{levins2006strategies,evans2012predictive}.


Inevitably, model building in biology leads to a key question that will, in turn, modify the outcomes achieved by any model: when is a model ``realistic'' enough \citep{stouffer2019all}? The answer to this question will depend on the purpose for which a model is built. Models that capture the effect of biotic interactions tend to fall into the category of ``demonstration models''. These types of models are often based on phenomenological descriptions of processes and have the general aim to show that the modeled principles are sufficient to produce the phenomena of interest \citep{evans2013simple}.  Demonstration models however, do not help decide whether the modelled principles are \textit{necessary} \citep{evans2013simple}. The task to decide the necessary principles and thus the answer to the question of when a model is realistic enough becomes the modeler's responsibility. In many cases, the answer to this question can appear arbitrary or solely determined by the predominant paradigm regarding the studied system \citep{holland2006comment,bascompte2006response,kokko2007ecogenetic,aladwani2019addition,mayfield2017higher,martyn2021identifying}.



Always favoring simple models in ecological and evolutionary studies can be problematic from two perspectives. First, the assumption that more complex models do not lead to general insights is seldomly tested. For example, most models that capture competitive interactions between plants have the implicit assumption that competitive effects between individuals are always additive and direct \citep{schoener1974some,freckleton2001predicting,kraft2015plant}.  However, when models and data collection were set up to capture non-additive effects of interactions between individuals of co-occurring species, the evidence overwhelmingly showed that including these levels of biotic complexity was necessary to capture plant interactions accurately \citep{mayfield2017higher,martyn2021identifying,lai2021non}.  Thus, in some cases, increasing complexity increases rather than hampers the general insights obtained from models of biotic interactions.

Second, failing to include necessary levels of complexity can hinder our ability to predict how natural communities will react to novel conditions. Predictions of how natural systems will behave in the future are inherently challenging \citep{sutherland2006predicting}. Nevertheless, ignoring heterogeneities at various levels can further complicate rather than simplify predictions  \citep{evans2012predictive}. For instance, demographic models tend to treat ecological and evolutionary dynamics separately, despite the general understanding that both processes are often intertwined \citep{macarthur1962some,kokko2007ecogenetic}. Ignoring eco-evolutionary feedbacks leads to predictions that are inconsistent with empirical data and produce counterintuitive results in novel conditions \citep{kokko2007ecogenetic}. Thus, the implicit assumption that good models should include as little as possible should be treated with caution in ecological and evolutionary contexts \citep{evans2013simple, kokko2007ecogenetic,abrams2001describing}.



\subsection*{Challenges and consequences of increasing realism}

Despite arguments in favor of increasing realism in models of biotic interactions, doing so remains a challenge in many ecological and evolutionary studies. One of those challenges arises from the lack of theoretical frameworks that allow incorporating intricate empirical observations into models \citep{abrams1983arguments}.  Such is the case of competition between pollinators that forage for the same resources \citep{thomson_importance_2020}. An overwhelmingly amount of empirical evidence shows that pollinators modify their foraging behavior in the presence of other foraging species \citep{morse_resource_1977,inouye_resource_1978,thompson_dynamics_2006,brosi_single_2013,briggs_competitive_2016}, however, incorporating these behavioral changes into population dynamics remains elusive \citep{thomson_importance_2020}. Furthermore, density-dependent responses could themselves depend on the abiotic conditions pollinators experience, as many studies have shown that environmental conditions can drastically change how pollinators behave while foraging \citep{heinrich_resource_1976,thomson_response_1987,cnaani_flower_2006,westphal_bumblebees_2006,briggs2018variation,classen2020specialization}. A coherent framework with which to incorporate both abiotic and biotic drivers into plant--pollinator interactions was lacking. To this end, in \textbf{Chapter 2} I develop a general framework to show how pollinator functional responses can be used to incorporate biotic and abiotic drivers into floral visitation rates. Furthermore, I show the empirical relevance of this framework by parameterizing different models that incorporated pollinator-pollinator interactions and environmental conditions when predicting observed data from a highly controlled foraging chamber experiment. Results from this chapter have substantial implications related to our understanding of how species loss and environmental change might affect mutualistic communities.


Another theoretical challenge emerges when alternative models to represent biotic interactions are used interchangeably. Such is the case of phenomenological models of plant competition, where more than one mathematical form can faithfully reproduce empirical data \citep{levine2009importance,godoy_phenology_2014,godoy_phylogenetic_2014,mayfield2017higher,bimler_accurate_2018}. The effect biotic and abiotic drivers have in model based predictions can be dramatically different due to uncertainty associated with phenomenological models \citep{jorgensen2001fundamentals,flora_structural_2011, aldebert_clement_community_2018}. To understand the interplay between uncertainty and abiotic complexity, in \textbf{Chapter 3} I introduce a mathematical and statistical framework to simultaneously explore how different phenomenological models of plant competition, environmental contexts, and parameter uncertainty determine predictions of species coexistence. Additionally, I use this framework to make predictions around a pairwise competition experiment between annual plants, where I show that the effect of abiotic conditions in predictions of species coexistence is not independent of the model formulation used to describe species interactions.

Finally, it is not only important to understand whether increasing realism changes predictions, but it is also essential to understand how. For instance, theoretical studies have shown that environmental fluctuations can substantially increase the levels of genetic diversity in populations that experience sexually antagonistic selection \citep{connallon2012general,connallon_evolutionary_2019,patten2010fitness,jordan2012potential}. However, there are no approaches that directly quantify how environmental fluctuations promote genetic diversity in populations that experience sexual conflict. Hence, in \textbf{Chapter 4} I adopt an ecological framework that allows the quantification of the relative contributions of environmental fluctuations using simulations to capture the effect of environmental fluctuations in evolutionary dynamics. In this chapter I show that incorporating environmental heterogeneity is essential to fully understand the effect of when and how sexually antagonistic selection can maintain genetic diversity.



\subsection*{Concluding remarks}

%Despite arguments in favor of increasing realism in models of biotic interactions, doing so remains a challenge in many ecological and evolutionary studies, which I address in the following section. Thus, it remains unclear whether increasing model realism is warranted for many natural systems.

In this thesis, I propose theoretical and statistical frameworks that allow increasing realism in models of biotic interactions with the aim to understand when higher levels of complexity are justified. Furthermore, I also explore the consequences of increasing model realism in predictions related to diversity maintenance at ecological and evolutionary scales. The individual chapters of this thesis are thematically broad as they are focused on different types of interactions and organisms, but all address in a different way the challenges and consequences of incorporating biotic and abiotic complexity in the study of biotic interactions.

%As part of this analysis, I also carried out a model comparison to show that models that include both biotic and abiotic drivers are needed to predict the number of visits a pollinator will make accurately.
%\section*{Concluding remarks}
%The individual chapters of this thesis are thematically broad but all address in a different way the consequences of increasing complexity in models of biotic interactions. With the exception of  I explore the consequences in terms of the coexistence of organisms. Through out this thesis I explored different ecological systems, with different types of interactions and species in them. However, the fundamental questions remains : what happens when add biological, environmental and mathematical complexity to the study of species interactions? Do they change our predictions?

% As scientists, narrative reasoning allows us to explore, at a high level, the possible trajectories that evolution may take.


%for discussion
%Theory reduces the apparent complexity of the natural world, because it cap- tures essential features of a system, provides abstracted char- acterizations, and makes predictions for as-yet unob
%No matter how much data one can obtain from social, bio- logical, and ecological systems, the multiplicity of entities and interactions among them means that we will never be able to predict many salient features of their structures and dynam- ics. To discover the underlying principles, mechanisms, and organization of complex adaptive systems and to develop a quantitative, predictive, conceptual framework ultimately requires the close integration of both theory and data.

\clearpage
\bibliographystyle{ecology_letters}
\bibliography{Bibliography.bib}

\end{document}
