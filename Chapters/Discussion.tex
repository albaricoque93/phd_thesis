\begin{refsection}
\chapter{Conclusion} % Main chapter title
\label{Conclusion}
\begin{flushright}{\slshape
    Where is the rest of the world? \\
    That is the question we must always ask about any model: \\
    where is the rest of the world?
    } \\ \medskip
    --- \textcite{levins2006strategies}
\end{flushright}

\bigskip

In this thesis, I show how to incorporate biotic and abiotic complexity in models of biotic interactions to increase model realism. Furthermore, I provide direct evidence that many models used to describe biotic interactions are oversimplistic since they fail to capture dynamics accurately by \textit{a priori} ignoring biotic and biotic factors. Throughout this thesis, I also show that increasing realism in models of biotic interactions has important repercussions on our understanding and predictions about the maintenance of diversity at ecological and evolutionary scales.

\section*{Summary of results}
In \autoref{Bee_foraging} I found that the abundance of co-foragers can substantially change the number of visits pollinators make. These results imply that it is necessary to account for the density of species other than the focal pair to characterize plant-pollinator interactions accurately. However, results from this chapter also show that the environmental context pollinators experience mediates density-dependent responses to co-foraging species. Thus, abiotic drivers can modify the number of visits made by pollinators through both density-independent and density-dependent responses. These two types of responses can cause the same environmental context to have opposite effects on floral visits. Such is the case of high resource abundance in our foraging experiment. Additionally, in this chapter I show that pollinators do not respond equally to all co-foraging species. Therefore the effects of biotic and abiotic drivers depend on the identity of the interacting species. Results from this chapter clearly show that including these levels of complexity in a model of floral visits is justified, despite the increasing number of parameters necessary to parameterize it. Since floral visitation is a good predictor of the strength of plant-pollinator interactions my results demonstrate that failing to account for biotic and abiotic complexity 

\section*{General Implications}
\subsection*{Model realism}
\subsection*{Diversity maintenance}
Models of mutualistic interactions are generally used to understand the number of species that can be maintainted in a community.

\section*{Future Directions}

\printbibliography
\end{refsection}
