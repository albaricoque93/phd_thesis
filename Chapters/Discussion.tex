\begin{refsection}
\chapter{Conclusion} % Main chapter title
\label{Conclusion}
\begin{flushright}{\slshape
    Where is the rest of the world? \\
    That is the question we must always ask about any model: \\
    where is the rest of the world?
    } \\ \medskip
    --- \textcite{levins2006strategies}
\end{flushright}

\bigskip

In this thesis, I show how to incorporate biotic and abiotic complexity in models of biotic interactions to increase model realism. Furthermore, I provide direct evidence that many models used to describe biotic interactions are oversimplistic since they fail to capture dynamics accurately by \textit{a priori} ignoring abiotic and biotic factors. Throughout this thesis, I also show that increasing realism in models of biotic interactions has important repercussions on our understanding and predictions about the maintenance of diversity at ecological and evolutionary scales.

\section*{Summary of results}
In \autoref{Bee_foraging} I found that the abundance of co-foragers can fundamentally change the number of visits pollinators make. These results imply that it is necessary to account for the density of species other than the focal pair to characterize plant-pollinator interactions accurately. However, results from this chapter also show that the environmental context pollinators experience mediates density-dependent responses to co-foraging species. Thus, abiotic drivers can modify the number of visits made by pollinators through both density-independent and density-dependent responses. These two types of responses can cause the same environmental context to have opposite effects on floral visits. Such is the case of high resource abundance in our foraging experiment. Additionally, in this chapter, I show that pollinators do not respond equally to all co-foraging species. Therefore the effects of biotic and abiotic drivers depend on the identity of the interacting species. Results from this chapter clearly show that including these levels of complexity in a model of floral visits is justified, despite the increasing number of parameters necessary to fit such a model. Since floral visitation is a good predictor of the strength of plant-pollinator interactions \citep{vazquez2005interaction, vazquez_strength_2012} my results demonstrate that failing to account for biotic and abiotic complexity can result in misleading estimations of the level of interdependence of animal and plant populations.


In \autoref{Bayesian_competition} I found that accounting for the abiotic context where interactions occur can fundamentally change predictions of species coexistence. While other studies have previously shown that predictions of coexistence between plant species can be context-dependent \citep{bimler_accurate_2018,lanuza_opposing_2018}, my results are the first to show that the estimated effect of the abiotic context depends on the model used to describe species interactions. Thus, my results show that incorporating abiotic complexity in models of biotic interactions is far from straightforward, as different phenomenological models can enhance or diminish its effect. Additionally, parameter uncertainty can further hinder the interpretation of the effect abiotic drivers have on predictions. For instance, predictions showed that in the \textit{woody} environment our focal species were unlikely to coexist, however, the species predicted to be competitively excluded varied across posterior draws. Therefore, my results show that robust predictions of species coexistence need to consider the abiotic context where interactions occur \textit{and} different sources of uncertainty associated with phenomenological models.

Finally, in \autoref{Coexistence_alleles} I found that environmental fluctuations can substantially increase the level of polymorphism in populations that experience sexually antagonistic selection. Perhaps most importantly, the results of this chapter show that environmental fluctuations can maintain disadvantageous alleles in a population by contributing positively to their growth rates when rare. However, the positive contributions of fluctuations depended on the pathway by which each allele was introduced into the population. Thus, I show that abiotic heterogeneity must be coupled with aspects of the evolutionary dynamics of the populations involved to maintain genetic diversity. This chapter highlights that not all types of abiotic drivers have the same effects on the populations involved. For instance, fluctuations in selection contributed positively to allele's invasion growth rates when fluctuations were positively correlated. In contrast, fluctuations in population sizes needed to be negatively correlated to have positive contributions. Therefore, my results show the importance of not only investigating \textit{if} environmental drivers change predictions but also \textit{how} they do it.


\section*{General Implications}

Increasing model realism can be achieved in multiple ways. One of them is to add independent variables to a model that represent previously ignored aspects of the real world \citep{orzack1993critical, evans2013simple}. For example, in \autoref{Bee_foraging}, I increased realism by adding variables that accounted for the densities of co-foragers to a model of floral visits. Other methods include adding a new link to variables already present or imposing bounds to some aspects of the model \citep{levins1993response}. For example, in \autoref{Bayesian_competition} I assumed unlimited growth was unrealistic and imposed abundance constraints when predicting species coexistence. Importantly, whether any of these methods increase the correspondence between model and phenomena of interest can not be evaluated \textit{a priori}. For instance, in  \autoref{Bee_foraging} it may have been the case that abiotic conditions had no effect on how a pollinator forages the presence of other species. Then, a model that included density-dependent effects to environmental conditions would have been over-parameterized. However, this assessment can only be done \textit{after} the fit a more complex model.


In this thesis, I show that models of biotic interactions aimed to make predictions regarding diversity maintenance tend to be oversimplistic. The implicit assumption that general insights can only be achieved with simple models \citep{holling1966strategy,may2019stability,roughgarden2018adaptive} has led to the automatic omission of biotic and abiotic complexity in many models of biotic interactions. However, scientific generality is not the same as mathematical generality \citep{levins1993response}.  For instance, to build a simple model in mathematical terms, we should include as little as possible \citep{orzack2012philosophy}. In contrast, building a simple model in biological terms might entail accounting for abiotic and biotic context-dependency.

A key simplyfing assumption in many models of biotic interactions is to focus on species pairs, even when multilple species are involved \citep{levine_beyond_2017}. However, many theoretical and empirical studies have shown that this assumption is likely a major oversimplification \citep{abrams1983arguments,billick_higher_1994,mayfield2017higher,letten_mechanistic_2019}. Various  biological mechanisms can cause individuals of the same or different species to modify the interaction between a focal pair, a phenomena broadly referred to as higher order interactions \citep{kleinhesselink2019mechanisms}. One of these mechanisms, interaction modification, arises when an intermediary species induces behavioral changes in one or both of the species in the focal pair, therby modyfing their interaction \citep{adler_general_1994, billick_higher_1994}. In \autoref{Bee_foraging}, I show that interaction modification is likely to occur in mutualistic communities by behavioral responses of pollinators to other foraging species. Evidence that shows higher order interactions have non-neglegible effect in natural communities continues to acumulate \citep{mayfield2017higher,martyn2021identifying,lai2020role,levine_beyond_2017}.


%biotic drivers and coexistence

%abiotic drivers and coexistence

\section*{Future Directions}



\printbibliography
\end{refsection}
