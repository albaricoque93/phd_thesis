\begin{refsection}
\chapter{Conclusion} % Main chapter title
\label{Conclusion}
\begin{flushright}{\slshape
    Where is the rest of the world? \\
    That is the question we must always ask about any model: \\
    where is the rest of the world?
    } \\ \medskip
    --- \textcite{levins2006strategies}
\end{flushright}

\bigskip

In this thesis, I show how to incorporate biotic and abiotic complexity in models of biotic interactions to increase realism. Furthermore, I provide direct evidence that many models used to describe biotic interactions are oversimplistic, as they fail to capture dynamics accurately by \textit{a priori} ignoring abiotic and biotic factors. Throughout this thesis, I also show that increasing realism in models of biotic interactions has important repercussions on our understanding and predictions about the maintenance of diversity at ecological and evolutionary scales.

\section*{Summary of results}
In \autoref{Bee_foraging}, I found that the abundance of co-foragers can fundamentally change the number of visits pollinators make. These results imply that it is necessary to account for the density of species other than the focal pair to characterize plant-pollinator interactions accurately. However, results from this chapter also show that the environmental context pollinators experience mediates density-dependent responses to co-foraging species. Thus, abiotic drivers can modify the number of visits made by pollinators through both density-independent and density-dependent responses. These two types of responses can cause the same environmental context to have opposite effects on floral visits. Such is the case of high resource abundance in our foraging experiment. Additionally, in this chapter, I show that pollinators do not respond equally to all co-foraging species. Therefore, the effects of biotic and abiotic drivers depend on the identity of the interacting species. Results from this chapter clearly show that including these levels of complexity in a model of floral visits is justified, despite the increasing number of parameters necessary to fit such a model. Since floral visitation is a good predictor of the strength of plant-pollinator interactions \citep{vazquez2005interaction, vazquez_strength_2012} my results demonstrate that failing to account for biotic and abiotic complexity can result in misleading estimations of the level of interdependence of animal and plant populations.


In \autoref{Bayesian_competition}, I found that accounting for the abiotic context where interactions occur can fundamentally change predictions of species coexistence. While other studies have previously shown that predictions of coexistence between plant species can be context-dependent \citep{bimler_accurate_2018,lanuza_opposing_2018}, my results are the first to show that the estimated effect of the abiotic context depends on the model used to describe species interactions. Thus, my results show that incorporating abiotic complexity in models of biotic interactions is far from straightforward, as different phenomenological models can enhance or diminish its effect. Additionally, parameter uncertainty can further hinder the interpretation of the effect abiotic drivers have on predictions. For instance, predictions showed that in the \textit{woody} environment our focal species were unlikely to coexist, however, the species predicted to be competitively excluded varied across posterior draws. Therefore, my results show that robust predictions of species coexistence need to consider the abiotic context where interactions occur \textit{and} different sources of uncertainty associated with phenomenological models.

Finally, in \autoref{Coexistence_alleles} I found that environmental fluctuations can substantially increase the level of polymorphism in populations that experience sexually antagonistic selection. Perhaps most importantly, the results of this chapter show that environmental fluctuations can maintain disadvantageous alleles in a population by contributing positively to their growth rates when rare. However, the positive contributions of fluctuations depended on the pathway by which each allele was introduced into the population. Thus, I show that abiotic heterogeneity must be coupled with aspects of the evolutionary dynamics of the populations involved to maintain genetic diversity. This chapter highlights that not all types of abiotic drivers have the same effects on the populations involved. For instance, fluctuations in selection contributed positively to allele's invasion growth rates when fluctuations were positively correlated. In contrast, fluctuations in population sizes needed to be negatively correlated to have positive contributions. Therefore, my results show the importance of not only investigating \textit{if} environmental drivers change predictions but also \textit{how} they do it.


\section*{General Implications}

Increasing model realism can be achieved in multiple ways. One of them is to add independent variables to a model that represent previously ignored aspects of the real world \citep{orzack1993critical, evans2013simple}. For example, in \autoref{Bee_foraging} I increased realism by adding variables that accounted for the densities of co-foragers to a model of floral visits. Other methods include adding a new link to variables already present or imposing bounds to some aspects of the model \citep{levins1993response}. For example, in \autoref{Bayesian_competition} I assumed unlimited growth was unrealistic and imposed abundance constraints when predicting species coexistence. Importantly, whether any of these methods increase the correspondence between model and phenomena of interest can not be evaluated \textit{a priori}. For instance, in  \autoref{Bee_foraging} it may have been the case that abiotic conditions had no effect on how a pollinator forages the presence of other species. Then, a model that included density-dependent responses to environmental conditions would have been over-parameterized. However, this assessment can only be done \textit{after} the fit a more complex model.


In this thesis, I show that models of biotic interactions aimed to make predictions regarding diversity maintenance tend to be oversimplistic. The implicit assumption that general insights can only be achieved with simple models \citep{holling1966strategy,may2019stability,roughgarden2018adaptive} has led to the automatic omission of biotic and abiotic heterogeneities in many models of biotic interactions. However, scientific generality is not the same as mathematical generality \citep{levins1993response}.  For instance, to build a simple model in mathematical terms, we should include as little as possible \citep{orzack2012philosophy}. In contrast, building a simple model in biological terms might entail accounting for abiotic and biotic dependency.

A key simplifying assumption in many models of biotic interactions is that species pairs is the relevant unit of study \citep{levine_beyond_2017}. However, many theoretical and empirical studies have shown that this assumption is likely a major oversimplification \citep{abrams1983arguments,billick_higher_1994,mayfield2017higher,letten_mechanistic_2019}. Various biological mechanisms can cause individuals of the same or different species to modify the interaction between a focal pair, a phenomenon broadly referred to as higher order interactions \citep{kleinhesselink2019mechanisms}. One of these mechanisms, interaction modification, arises when an intermediary species induces behavioral changes in one or both of the species in the focal pair, thereby modifying their interaction \citep{adler_general_1994, billick_higher_1994}. In \autoref{Bee_foraging}, I show that interaction modifications are likely to occur in mutualistic communities by behavioral responses of pollinators to other foraging species. Evidence that shows higher order interactions have non-negligible effect in natural communities continues to accumulate \citep{mayfield2017higher,martyn2021identifying,lai2020role,levine_beyond_2017}. Therefore, ignoring biotic complexity in favor of mathematical simplicity can be detrimental to understanding how multispecies communities are assembled and maintained.

 The environmental dependence of biotic interactions has broad empirical and theoretical support \citep{callaway_positive_2002,chamberlain_how_2014,lanuza_opposing_2018,chesson2000mechanisms,tylianakis2008global,bimler_accurate_2018}. Furthermore, the importance of heterogeneous environments in the maintenance of diversity has been shown at ecological \citep{amarasekare2003competitive,kneitel2004trade} and evolutionary scales \citep{connallon_evolutionary_2019,ellner1994role,dean2005protecting}. Yet, most models used to make make predictions regarding diversity maintenance tend to keep changes in biotic interactions due to environmental effects implicit or treat them as constant (but see \citet{bimler_accurate_2018}, or \citet{connallon_evolutionary_2019}). In \autoref{Bayesian_competition} and \autoref{Coexistence_alleles} I address two different challenges when accounting for environmental dependency in models of biotic interactions: the interplay between model uncertainty and environmental heterogeneity, and the interpretation of environmental effects in model predictions. While in natural systems, as opposed to experimental systems or simulations, it is challenging to determine exactly what variables constitute ``the environment'', an inability to account for abiotic dependence prevents ecologists and evolutionary biologists to correctly identify the drivers of diversity maintenance \citep{freckleton2009measuring, connallon_evolutionary_2019}.


%biotic drivers and coexistence

%abiotic drivers and coexistence

\section*{Future Directions}

While it might be tempting to argue that increasing model realism is a Sisyphean endeavor, there are limits to the level of complexity that can be added to a model without losing its usefulness. Where those limits are and whether a model is realistic enough will depend on the state of science at the time \citep{levins1993response}. In this thesis, I argue that automatically excluding biotic and abiotic factors from models of biotic interactions in favor of mathematical simplicity obstacles our understanding of how natural systems operate. The level of realism needed to describe biotic interactions accurately will undoubtedly depend on the system studied, as not all types of interactions are equally likely to be affected by the biotic and abiotic context they are embedded in \citep{chamberlain_how_2014}. Nonetheless, scientific progress requires that we acknowledge and explore this complexity. This has been and continues to be done by studies that compare models of varying levels of complexity to understand when its inclusion is warranted \citep{lai2020role,martyn2021identifying,bimler_accurate_2018,weiss2021disentangling}, studies that investigate the biological rationale for why certain levels of complexity should be included in models \citep{abrams1983arguments,abrams_nature_2000,letten_mechanistic_2019,stouffer2021hidden,aladwani2019addition}, and studies that show how to account for unmeasured biotic and abiotic drivers \citep{d2018translucent, song_towards_2020}.

Finally, a key aspect missing from most ecological and evolutionary research focused on diversity maintenance is the bidirectional link between organisms and the environment. Extensive research, including this thesis, has been done to disentangle the effects of abiotic and biotic factors on the performance of organisms. The reciprocal phenomenon, the reaction and evolution of the environment in response to the organisms embedded in it, is generally ignored in many ecological and evolutionary studies \citep{levins_dialectics_1980,laland1999evolutionary}. However, including the feedback loop between organism and environment is essential to fully understand the interplay between abiotic and abiotic variables in natural communities \citep{callaway2007positive,hastings2007ecosystem}. Additionally, theoretical studies suggest that this feedback process can fundamentally change predictions related to diversity maintenance in natural communities \citep{kylafis_niche_2011,kylafis2008ecological}. Thus, subsequent efforts should examine how biotic and abiotic complexity jointly modifies the responses of organisms and the environment. This thesis offers some theoretical tools for ecologists and evolutionary biologists to explore the first part of this puzzle.
\printbibliography
\end{refsection}
