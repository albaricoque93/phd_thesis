\begin{refsection}
\chapter{Conclusion} % Main chapter title
\label{Conclusion}
\begin{flushright}{\slshape
    Where is the rest of the world? \\
    That is the question we must always ask about any model: \\
    where is the rest of the world?
    } \\ \medskip
    --- \textcite{levins2006strategies}
\end{flushright}

\bigskip

In this thesis, I show how to incorporate biotic and abiotic complexity in models of biotic interactions to increase model realism. Furthermore, I provide direct evidence that many models used to describe biotic interactions are oversimplistic since they fail to capture dynamics accurately by \textit{a priori} ignoring abiotic and biotic factors. Throughout this thesis, I also show that increasing realism in models of biotic interactions has important repercussions on our understanding and predictions about the maintenance of diversity at ecological and evolutionary scales.

\section*{Summary of results}
In \autoref{Bee_foraging} I found that the abundance of co-foragers can fundamentally change the number of visits pollinators make. These results imply that it is necessary to account for the density of species other than the focal pair to characterize plant-pollinator interactions accurately. However, results from this chapter also show that the environmental context pollinators experience mediates density-dependent responses to co-foraging species. Thus, abiotic drivers can modify the number of visits made by pollinators through both density-independent and density-dependent responses. These two types of responses can cause the same environmental context to have opposite effects on floral visits. Such is the case of high resource abundance in our foraging experiment. Additionally, in this chapter, I show that pollinators do not respond equally to all co-foraging species. Therefore the effects of biotic and abiotic drivers depend on the identity of the interacting species. Results from this chapter clearly show that including these levels of complexity in a model of floral visits is justified, despite the increasing number of parameters necessary to parameterize it. Since floral visitation is a good predictor of the strength of plant-pollinator interactions \citep{vazquez2005interaction, vazquez_strength_2012} my results demonstrate that failing to account for biotic and abiotic complexity can result in misleading estimations of the level of interdependence of animal and plant populations.


In \autoref{Bayesian_competition} I found that accounting for the abiotic context where interactions occur can fundamentally change predictions of species coexistence. While other studies have previously shown that predictions of coexistence between plant species can be context-dependent \citep{bimler_accurate_2018,lanuza_opposing_2018}, my results are the first to show that the estimated effect of the abiotic context depends on the model used to describe species interactions. Thus, my results show that incorporating abiotic complexity in models of biotic interactions is far from straightforward, as different phenomenological models can enhance or diminish its effect. Additionally, parameter uncertainty can further hinder the interpretation of the effect abiotic conditions have on predictions. For instance, predictions showed that in the \textit{woody} environment our focal species were unlikely to coexist, however, the species predicted to be competitively excluded varied across posterior draws. Therefore my results show that robust predictions of species coexistence need to consider the abiotic context where interactions occur \textit{and} different sources of uncertainty associated with phenomenological models.

Finally, in \autoref{Coexistence_alleles} I found that environmental fluctuations can substantially increase the level of polymorphism in populations that experience sexually antagonistic selection. Perhaps most importantly, the results of this chapter show that environmental fluctuations can maintain disadvantageous alleles in a population by contributing positively to their growth rates when rare. However, the positive contributions of fluctuations depended on the pathway by which each allele was introduced into the population. Thus, I show that abiotic complexity must be coupled with aspects of the evolutionary dynamics of the populations involved to maintain genetic diversity. This chapter highlights that not all types of abiotic drivers have the same effects on the populations involved. For instance, fluctuations in selection contributed positively to allele's invasion growth rates when fluctuations were positively correlated. In contrast, fluctuations in population sizes needed to be negatively correlated to have positive contributions. Therefore, my results show the importance of not only investigating \textit{if} environmental drivers change predictions but also \textit{how} they do it.


\section*{General Implications}
%model realism
Model building is a process embeded in the larger process of scientific research. However, models themselves are not theories. Biology is full of false dychotomies 

%biotic drivers and coexistence

%abiotic drivers and coexistence

\section*{Future Directions}



\printbibliography
\end{refsection}
