\begin{refsection}
\chapter{Introduction} % Main chapter title
\label{Intro}

Interactions between organisms underpin the persistence of almost all life forms on Earth \citep{lawton1999there}. Furthermore, large body of work has shown that biotic interactions determine emergent properties of natural systems, such as stability \citep{may1972will, wootton2016many,song2018will}, resilience \citep{capdevila2021reconciling}, ecosystem functioning \citep{turnbull2013coexistence,godoy2020excess}, and the coexistence of multiple species \citep{chesson2000mechanisms,saavedra2017structural}. Unsurprisingly, numerous ecological and evolutionary concepts revolve around the reciprocal forces that organisms exert on each other \citep{gause_experimental_1934, macarthur1967limiting, hillerislambers2012rethinking,chase2009ecological, thompson1999evolution}.
%concepts: competitive exclusion, limiting similarity, community assembly, ecological niche, co-evolution

The study of biotic interactions often requires the use of mathematical models to represent them \citep{maynard1978models}. Mathematical descriptions of interactions are ``useful fictions'' \citep{box2011statistical} in a twofold manner. First, they create a description of how organisms that coincide in space and time reciprocally affect each other. Almost all known types of interactions can be somewhat accurately described in the form of mathematical expressions; mutualistic interactions, competition, parasitism, commensalism, and even the co-evolution of species. Second, they are practical tools with which to make predictions beyond the phenomena they describe. For instance.

However, models that capture the effect of biotic interactions, are abstractions of reality. It is rare that we know the exact equations governing a system or the full set of biotic and abiotic factors (song). Our abstractions always reflect choices.  A common assumption in ecological en evolutionary models, is that in order to achieve general insight, we should favour simple models. Indeed there is a general belief in ecology and evolution that a good model should include as little as possible.


The simplifying assumptions made to represent biotic interactions can dramatically impact model-based predictions. For instance, the elementary pair-wise interactions between species have been studied extensively.

Thus, When is relaxing the simplifying assumptions in models of  of biotic interactions necessary? Theoretical studies typically make two critical assumptions that do not hold in real communities, thus limiting their applicability.


In this thesis, I add another criteria to that, when it fundamentallyl.

%The first issue encountered in the development of a model is that all models are abstractions; it is obviously impossible to include every aspect of the real world in any model. Once a modeller recognizes that they cannot include all variables in a model, they have to make decisions about which to retain in the model and which to omit [6]. Such decisions will modify the outcomes achieved by a model and the process should be guided by the modellers’ objectives

%Species interactions do not present themselves to us. We often choose to represent them as parameters in a model, and so they become abstractions of reality. Our abstractions always reflect choices. It is rare that we know the exact equations governing a system or the full set of biotic and abiotic factors (song), therefore our approach to study species interactions and their consequences
%But organisms do not react to the environment as a whole, rather they react to some aspect of the environment
%\printbibliography

\section*{Introducing biotic drivers}

I investigate precisely this and explore how


\section*{ Exploring uncertainty}


\section*{Non-constant environments}



\section*{Concluding remarks}
The individual chapters of this thesis are thematically broad but all address in a different way the consequences of increasing complexity in models of biotic interactions. With the exception of \autoref{Bee_foraging} I explore the consequences in terms of the coexistence of species.

 changes our understanding of the system. Through out this thesis I explored different ecological systems, with different types of interactions and species in them. However, the fundamental questions remains : what happens when add biological, environmental and mathematical complexity to the study of species interactions? Do they change our predictions?

 As scientists, narrative reasoning allows us to explore, at a high level, the possible trajectories that evolution may take.

\printbibliography
\end{refsection}
