\begin{refsection}
\chapter{General Introduction} % Main chapter title
\label{Intro}
\begin{flushright}{\slshape
    Things are similar: this makes science possible \\
    Things are different: this makes science necessary
    } \\ \medskip
    --- \textcite{levins_dialectics_1980}
\end{flushright}

\bigskip

Interactions between organisms underpin the persistence of almost all life forms on Earth \citep{lawton1999there}. Furthermore, a large body of work has shown that biotic interactions determine emergent properties of natural systems, such as stability \citep{may1972will, wootton2016many,song2018will}, resilience \citep{capdevila2021reconciling}, ecosystem functioning \citep{turnbull2013coexistence,godoy2020excess}, and the coexistence of multiple species \citep{chesson2000mechanisms,saavedra2017structural}. Unsurprisingly, numerous ecological and evolutionary concepts revolve around the effects that organisms exert on each other \citep{gause_experimental_1934,macarthur1967limiting,thompson1999evolution, hillerislambers2012rethinking, chase2009ecological,thompson2014interaction}.

From their origins as natural sciences, the disciplines of ecology and evolution have shifted from a descriptive towards a more predictive and quantitative approach \citep{holling1966strategy,pickett1980non,simberloff2004community,marquet2014theory,lassig2017predicting,rossberg2019let}. This shift brought with it the use of mathematical models to describe natural phenomena, such as the effects species have on each other \citep{holling1966strategy,levins1966strategy,maynard1978models,servedio2014not}. Mathematical descriptions of interactions are ``useful fictions'' \citep{box2011statistical} in a twofold manner. First, they create a description of how organisms that coincide in space and time affect each other. Almost all known types of interactions can be described in the form of mathematical expressions that faithfully reproduce features of the observed data \citep{volterra1926fluctuations,holling1959some,holt1977predation,adler2018competition,wood1999super,holland2002population,vazquez2005interaction,stouffer2021hidden}. Second, models are practical tools with which to make predictions beyond the phenomena they describe and thus provide general insights into how natural systems operate \citep{sutherland2006predicting,stouffer2019all}.


\subsection*{Model tradeoffs}
Models that capture the effect of biotic interactions are abstractions of reality, and abstractions always reflect choices \citep{levins2006strategies}. Building models that include all aspects of reality is not only impractical but also unfeasible.  Therefore, ecologists and evolutionary biologists have to continuously make choices regarding which variables to include in a model and which to omit \citep{odenbaugh2005idealized}. A common assumption when building models is that to achieve general insights, we should favor simple models \citep{evans2013simple}. Indeed there is a general belief in ecology and evolution that a general model should include as little as possible \citep{holling1966strategy,may2019stability,roughgarden2018adaptive}. This belief is often rooted in an implicit philosophical stance that one can not simultaneously maximize generality (i.e., models that apply to more than one system) and realism (i.e., models that produce accurate predictions for a given system) \citep{levins1966strategy,levins1993response}.


Inevitably, model building in biology leads to a key question that will, in turn, modify the outcomes achieved by any model: when is a model ``realistic'' enough \citep{stouffer2019all}? The answer to this question will depend on the purpose for which a model is built \citep{odenbaugh2005idealized,levins2006strategies}. The classification of biological models and their purposes have been and continue to be widely debated \citep{holling1966strategy,may2019stability,lewontin1963models,levins1966strategy,orzack1993critical,levins1993response,odenbaugh2005idealized,weisberg2006forty,evans2013simple}. Overall, it is generally recognized that the purposes of different biological models fall on a continuum \citep{levins1993response,evans2013simple,servedio2014not}. On one end of this continuum are models that aim to understand and identify general principles (called strategic models by \citet{holling1966strategy} and \citet{may2019stability}, or a minimal model of ideas by \citet{roughgarden2018adaptive}). On the other end are models that aim to make detailed quantitative predictions (also called tactical models by \citet{holling1966strategy} or synthetic models by \citet{roughgarden2018adaptive}). The tradeoffs between generality, realism, and precision at each end of the spectrum have sparked extensive debate among biologists \citep{levins1966strategy,orzack1993critical,levins1993response,weisberg2006forty}.

Models that capture the effect of biotic interactions tend to fall in the spectrum under the category of ``demonstration models'', as first defined by \citet{crick1988mad} and later by \citet{evans2013simple}. These types of models are often based on phenomenological descriptions of processes and have the general aim to show that the modeled principles are sufficient to reproduce some phenomena of interest \citep{crick1988mad,evans2013simple}.  Demonstration models, however, do not help decide whether the modelled principles are \textit{necessary} \citep{evans2013simple}. The task to decide the necessary principles and thus the answer to the when a model is realistic enough becomes the modeler's responsibility. In many cases, the answer to this question can appear arbitrary or solely determined by the dominant paradigm regarding the studied system. For example, mutualistic interactions between two species can be described by a simple model that assumes a linear functional response \citep{bascompte2006asymmetric}, or by a more realistic model that incorporates saturating effects \citep{holland2002population}.  The choice between these two models has substantial implications for predictions related to the coexistence of species and the assembly of communities \citep{holland2002population}. However, there is no consensus on which representation to favor, as the choice is usually defined by the modeler's particular school of thought and mathematical convince \citep{holland2006comment, bascompte2006response}.

\subsection*{The perils of simple models}
A stance that always favors simple models in ecological and evolutionary studies can be problematic from two perspectives. First, the assumption that more complex models do not lead to general insights is seldom tested. For example, most models that capture competitive interactions between plants have the implicit assumption that competitive effects between individuals are always additive and direct \citep{schoener1974some,freckleton2001predicting,kraft2015plant}.  However, when models were set up to capture non-additive effects of interactions between individuals of co-occurring species, the evidence overwhelmingly showed that including these levels of biotic complexity was necessary to capture plant interactions accurately \citep{mayfield2017higher,martyn2021identifying,lai2021non}.  Thus, in some cases, increasing complexity increases rather than hampers the general insights obtained from models of biotic interactions.

Second, failing to include necessary levels of complexity can hinder our ability to predict how natural communities will react to novel conditions. Predictions of how natural systems will behave in the future are inherently challenging \citep{sutherland2006predicting}. Nevertheless, ignoring heterogeneities at various levels can further complicate rather than simplify predictions \citep{d2018translucent}. For instance, demographic models tend to treat ecological and evolutionary dynamics separately, despite the general understanding that both processes are often intertwined \citep{macarthur1962some,kokko2007ecogenetic}. Ignoring eco-evolutionary feedbacks leads to predictions that are inconsistent with empirical data and produce counterintuitive results in novel conditions \citep{kokko2007ecogenetic}. Thus, the implicit assumption that good models should include as little as possible should at least be treated with caution in ecological and evolutionary contexts.



\subsection*{Challenges and consequences of increasing realism}

Despite arguments in favor of increasing realism in models of biotic interactions, doing so remains a challenge in many ecological and evolutionary studies. One of those challenges arises from the lack of theoretical frameworks that allow incorporating intricate empirical observations into models \citep{abrams1983arguments,abrams2001describing}.  Such is the case of competition between pollinators that forage for the same resources \citep{thomson_importance_2020}. An overwhelming amount of empirical evidence shows that pollinators modify their foraging behavior in the presence of other foraging species \citep{morse_resource_1977,inouye_resource_1978,thompson_dynamics_2006,brosi_single_2013,briggs_competitive_2016}; however, models that incorporate these behavioral changes into population dynamics remain scarce \citep{thomson_importance_2020}. Furthermore, density-dependent responses could themselves depend on the abiotic conditions pollinators experience, as many studies have shown that environmental conditions can drastically change how pollinators behave and interact with plant species \citep{heinrich_resource_1976,thomson_response_1987,cnaani_flower_2006,westphal_bumblebees_2006,briggs2018variation,classen2020specialization}. A coherent framework with which to incorporate both abiotic and biotic drivers into plant--pollinator interactions was lacking. To this end, in \autoref{Bee_foraging} I develop a general framework to show how pollinator functional responses can be used to incorporate biotic and abiotic drivers into models of floral visitation rates. Furthermore, I show the empirical relevance of this framework by parameterizing different models of varying complexity that incorporate pollinator--pollinator interactions and environmental conditions when predicting observed data from a highly controlled foraging chamber experiment. Results from this chapter provide important insights related to our understanding of how species loss and environmental change might affect mutualistic communities.


Another theoretical challenge emerges when alternative models to represent biotic interactions are used interchangeably. Such is the case of phenomenological models of plant competition, where more than one mathematical form can faithfully reproduce empirical data \citep{levine2009importance,godoy_phenology_2014,godoy_phylogenetic_2014,mayfield2017higher,bimler_accurate_2018}. The effect biotic and abiotic drivers have on model based predictions can be dramatically different due to uncertainty associated with phenomenological models \citep{jorgensen2001fundamentals,flora_structural_2011, aldebert2018community}. To understand the interplay between uncertainty and abiotic complexity, in \autoref{Bayesian_competition} I introduce a mathematical and statistical framework to simultaneously explore how different phenomenological models of plant competition, environmental context, and parameter uncertainty impact predictions of species coexistence. Additionally, I use this framework to make predictions around a pairwise competition experiment between annual plants, where I show that the effect of abiotic conditions in predictions of coexistence outcomes is not independent of the model formulation used to describe species interactions.

Finally, even when existing studies show that increasing model realism is warranted, understanding exactly how the addition of complexity changes predictions remains a challenge. For instance, theoretical and empirical studies have shown that environmental fluctuations can substantially increase the levels of genetic diversity in populations that experience sexually antagonistic selection \citep{connallon2012general,connallon_evolutionary_2019, glaser2021sexual}. However, there are no approaches that directly quantify \textit{how} abiotic heterogeneity promotes the maintenance of genetic diversity in populations that experience sexual conflict. Hence, in \autoref{Coexistence_alleles} I adopt an ecological framework to explicitly quantify the contributions of fluctuations in population sizes and selection to alleles' growth rates when rare using simulations. I show that environmental fluctuations can help maintain genetic variance in a population by allowing disadvantageous alleles to have positive invasion growth rates,  but that their effect depends on the pathway by which each allele is introduced to the population.


\subsection*{Concluding remarks}

%Despite arguments in favor of increasing realism in models of biotic interactions, doing so remains a challenge in many ecological and evolutionary studies, which I address in the following section. Thus, it remains unclear whether increasing model realism is warranted for many natural systems.

In this thesis, I propose theoretical and statistical frameworks that allow increasing realism in models of biotic interactions with the aim of understanding when higher levels of complexity are justified. Furthermore, I also explore the consequences of increasing model realism in predictions related to diversity maintenance at ecological and evolutionary scales. The individual chapters of this thesis are thematically broad as they are focused on different types of interactions and organisms, but all address in a different way the challenges and consequences of incorporating biotic and abiotic complexity in the study of biotic interactions.


\printbibliography
\end{refsection}
