\begin{refsection}
\chapter{Introduction} % Main chapter title
\label{Intro}

The reciprocal forces organism exert one another create emergent properties in natural communities. Indeed, a large body of work has shown that the way species affect one another determinde key properies such as stability, resilience, productivity, and the coexistence of multiple species. This is because
%Are these abstractions, what we understand as interactions, meaningful or obfuscatory?

Species interactions do not present themselves to us. We often choose to represent them as parameters in a model, and so they become abstractions. Our abstractions always reflect choices.
%But organisms do not react to the environment as a whole, rather they react to some aspect of the environment
%\printbibliography


Ecological and evolutionary research has begun to incorporate the complexity in our the study of species interactions, I am part of that research. Through out this thesis I explored different ecological systems, with different types of interactions and species in them. However, the fundamental questions remains : what happens when add biological, environmental and mathematical complexity to the study of species interactions? Do they change our predictions?

\end{refsection}
