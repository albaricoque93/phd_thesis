\begin{refsection}
\chapter{The context dependency of pollinator interference: how environmental conditions and co-foraging species impact floral visitation} % Main chapter title
\label{Bee_foraging}

\noindent Alba Cervantes-Loreto\textsuperscript{1}, Carolyn A.\ Ayers \textsuperscript{2}, Emily K.\ Dobbs\textsuperscript{2},Berry J.\ Brosi\textsuperscript{3}, Stouffer Daniel B.\textsuperscript{1}

\begin{enumerate}
    \item Centre for Integrative Ecology, School of Biological Sciences, University of Canterbury, New Zealand
    \item Department of Environmental Sciences, Emory University, Atlanta, Georgia USA
    \item University of Washington, Department of Biology, Seattle USA
\end{enumerate}

\section*{Abstract}
Animals often change their behavior in the presence of other species and the environmental context they experience, and these changes can substantially modify the course their populations follow. In the case of animals involved in mutualistic interactions, it is still unclear how to incorporate the effects of these behavioral changes into population dynamics. We propose a framework for using pollinator functional responses to examine the roles of pollinator--pollinator interactions and abiotic conditions in altering the times between floral visits of a focal pollinator. We then apply this framework to a unique foraging experiment with different models that allow resource availability and sub-lethal exposure to a neonicotinoid pesticide to modify how pollinators forage alone and with co-foragers. We found that all co-foragers interfere with the focal pollinator under at least one set of abiotic conditions; for most species, interference was strongest at higher levels of resource availability and with pesticide exposure. Overall our results highlight that density-dependent responses are often context-dependent themselves.

\section*{Introduction}
Interactions between pollinators have been extensively documented and described by ecologists \citep{mallinger2017managed, thomson_importance_2020}. For eusocial insects like some bees and bumblebees, the presence of other species has been shown to drive resource partitioning due to active avoidance \citep{morse_resource_1977, inouye_resource_1978}, change pollinator foraging efforts  \citep{thomson_detecting_2006}, and to promote short-term floral specialization \citep{brosi_single_2013, briggs_competitive_2016}. However, fundamental gaps remain regarding the consequences of pollinator--pollinator interactions in natural communities, mainly because of the complexity of linking the effects of the interaction to population dynamics \citep{thomson_importance_2020}.


One of the empirical challenges in understanding interactions between pollinators is that environmental conditions can drastically change how pollinators behave and interact with conspecifics and other species. For instance, plant--pollinator interactions tend to be contingent on the external conditions pollinators experience \citep{heinrich_resource_1976,cnaani_flower_2006,briggs_variation_2018}. High resource availability---measured in flower density or nectar volume---has been shown to decrease the duration of foraging trips for bumblebees \citep{westphal_foraging_2006} and increase floral visits \citep{thomson_response_1987, thomson_effects_1988}. Insect pollinators also show changes in their interactions with plants due to temperature; higher temperatures have been documented to shorten the time spent on individual flowers relative to low temperatures for bumblebees \citep{heinrich_energetics_1972} and to promote floral specialization within an elevation gradient \citep{classen2020specialization}. Hence, studying the context in which interactions occur is as important as studying the interactions themselves.


In contrast, a theoretical challenge is incorporating the behavioral changes driven by the presence of other foraging pollinators, henceforth co-foragers, into population dynamics. Pollinator functional responses, which describe how consumption rates vary with the abundance of individuals of another population \citep{holland_population_2001}, are key to how pollinator and plant populations are linked to each other. When pollinators modify their behavior due to the presence of other foraging species, it echoes observations in which predators' consumption rates vary because of ``interference'': time spent engaging in encounters with other predators instead of feeding \citep{beddington_mutual_1975,deangelis_emergence_2006,skalski_functional_2001}.

Overt interference between pollinators is thought to occur only for very specific groups of pollinators that present aggressive behavior, such as stingless bees that can recruit in large numbers and inflict serious damage to their competitors \citep{lichtenberg_olfactory_2011}. Nonetheless, the presence of other foragers could have the same phenomenological effect as overt interference---from a functional response perspective---as long as it decreases the visitation rates of a focal individual. Importantly, the presence of other pollinator species can also increase visitation rates \citep[e.g.][]{greenleaf_wild_2006}. Overall, whether or not the presence of other species leads to measurable differences in the rate of floral visits has equivocal experimental evidence: some studies report an increase in visits and pollination efficiency when more than one species is present \citep{frund_bee_2013} whereas others find an overall decrease in foraging activity \citep{roubik_competitive_1978, thomson_detecting_2006, thomson_importance_2020}. That the effect of varying pollinator abundances is context dependent could potentially explain the equivocal evidence found across the literature.


Fully incorporating pollinator behavioral changes into population dynamics is a laborious and challenging effort since it not only requires quantifying functional responses of the populations involved but the numerical responses as well \citep{revilla2015numerical, abrams_nature_2000}. Nonetheless, since interactions and vistation are a necessary precursor to a quantifiable numerical response, a good starting place is to determine how biotic and abiotic factors can be incorporated into pollinator's foraging rates. In this study, we therefore show how plant--pollinator functional responses can be used to incorporate the effects of environmental conditions and pollinator--pollinator interactions into floral visitation rates. We first introduce a novel framework that examines a simple response variable: the time a pollinator takes between floral visits. We then use our functional response framework to quantify the effects of pollinator--pollinator interactions under different environmental conditions in a highly controlled foraging-chamber experiment. Our experiments simultaneously modified varying levels of resource availability, sub-lethal exposure to a neonicotinoid pesticide, and co-foraging pollinator richness and abundance. We parameterize different models that incorporate pollinator--pollinator interactions and environmental conditions when predicting observed times between floral visits. Finally, we use these model fits to show that pollinator--pollinator interactions and their effects on focal pollinators are strongly determined by abiotic conditions.

\end{refsection}
