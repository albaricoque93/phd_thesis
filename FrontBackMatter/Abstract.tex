\pdfbookmark[1]{Abstract}{Abstract} % Bookmark name visible in a PDF viewer
\chapter*{Abstract}
\small
Interactions between organisms have a central role in the study of ecological and evolutionary dynamics. The study of biotic interactions usually requires the use of models and simplifying assumptions about reality, since it is impossible to include every aspect of the real world in any model.  Choices like the number of species that can alter the interaction between a focal pair, what abiotic variables constitute the environment, and even what type of mathematical formulation to use to capture the system's dynamics are common yet implicit in many ecological and evolutionary models. However, simplifying assumptions can lead to ignoring important heterogeneities at various levels, which could dramatically change model-based predictions. In this thesis, I explore with theoretical and empirical tools how relaxing simplifying assumptions in models of interactions between organisms change predictions related to diversity maintenance at ecological and evolutionary scales. Throughout my thesis, I focus on different types of interactions and organisms, and propose mathematical and statistical frameworks to incorporate biotic and abiotic variables, as well as different sources of uncertainty in our representations of biotic interactions.


In Chapter 2 I explore how the presence of multiple species and different environmental contexts change the strength of plant-pollinator interactions. I propose a framework for using pollinator functional responses to examine the roles of pollinator-pollinator interactions and abiotic conditions in altering the times between floral visits of a focal pollinator. I show that while density-dependent responses can substantially change the predicted number of visits a pollinator makes, they also strongly depend on the abiotic context pollinators experience. In Chapter 3  I explore how incorporating different sources of uncertainty changes our predictions of species coexistence. I do this by simultaneously exploring how different model formulations, environmental contexts, and parameter uncertainty change the probability of predicting coexistence in an experimental system. I provide direct evidence that predictions of species coexistence are likely to change given the models used to quantify density-dependence as well as a theoretical framework to explore predictions made with different models. Finally, in Chapter 4  I adopted an ecological framework to examine evolutionary dynamics of sexually antagonistic alleles through the same lens as the coexistence of competing species. I show that incorporating environmental fluctuations can substantially increase the amount of genetic diversity in a population under sexually antagonistic selection. Overall, the results of this thesis show that the assumptions adopted by some ecological and evolutionary models tend to be oversimplifying. This thesis also provides the tools for ecologists and evolutionary biologists to explore a more complex representation of biotic interactions
