\pdfbookmark[1]{Abstract}{Abstract} % Bookmark name visible in a PDF viewer
\chapter*{Abstract}
\small
Interactions between organisms give rise to emergent properties of natural systems. This underpins the ubiquity of biotic interactions in the study of ecological and evolutionary dynamics. The representation of biotic interactions often requires models and simplifying assumptions since it is impossible to account for all aspects of the world in a single model. Critical choices, such as the number of species that can alter the interaction between a focal pair or which abiotic variables constitute the environment, are necessary when building ecological and evolutionary models. Such simplifying assumptions inevitably lead to the omission of heterogeneities at various levels. Complexity that is unaccounted for can, in turn, make the relationships between organisms appear noisy and fundamentally change model-based predictions. Despite this, ecological and evolutionary studies often lack appropriate frameworks that allow the inclusion of different levels of complexity in representations of biotic interactions. Thus, it is unclear whether including more realistic assumptions is warranted for the vast majority of natural systems. In this thesis, I explore how incorporating complexity as abiotic and biotic modifiers, as well as different sources of uncertainty, reveals potential explanatory generalities in natural systems. I also explore how accounting for these variables changes predictions related to the maintenance of diversity at ecological and evolutionary scales. Throughout this thesis, I focus on different types of interactions and organisms and propose mathematical and statistical frameworks that can be used beyond the studied systems.


In \autoref{Bee_foraging}, I explore how the presence of multiple species and different environmental contexts change the strength of plant-pollinator interactions. I propose a framework for using pollinator functional responses to examine the role of pollinator-pollinator interactions and abiotic conditions in altering the time between floral visits of a focal pollinator. I show that both density dependent responses and abiotic conditions are necessary to explain the number of visits a pollinator makes. In \autoref{Bayesian_competition},  I explore how incorporating different sources of uncertainty changes predictions of species coexistence. I do this by simultaneously exploring how different model formulations, environmental contexts, and parameter uncertainty change the probability of predicting coexistence in a pairwise competition experiment of annual plants. I provide direct evidence that predictions of species coexistence are likely to change given the models used to quantify density-dependence. I also provide a theoretical framework to explore predictions made with different models. Finally, in \autoref{Coexistence_alleles}, I adopt an ecological framework to examine the evolutionary dynamics of sexually antagonistic alleles. I show that environmental fluctuations can substantially increase the amount of genetic diversity in a population under sexually antagonistic selection by contributing positively to allele's invasion growth rates. Overall, the results of my thesis show that the assumptions adopted by some ecological and evolutionary models tend to be oversimplifying. Here, I provide tools for ecologists and evolutionary biologists to explore  more realistic representations of biotic interactions as well as their consequences for diversity maintenance.
