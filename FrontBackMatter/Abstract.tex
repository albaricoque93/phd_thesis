\pdfbookmark[1]{Abstract}{Abstract} % Bookmark name visible in a PDF viewer
\chapter*{Abstract}
\small
	Species interactions have non random consequences in natural communities, thus they have a central role in our explanations of diversity maintenance. However,the magnitude, sign  and outcome of species interactions have shown repeatedly to be context dependent. In order to disentangle the role of contingency in nature, we must first explore the noise that arises from the way we study species interactions. This thesis is focused in exploring with theoretical and empirical approaches how model formulation, parametrization and reductionism alter our ecological predictions. Chapter one explores the structural sensitivity of species coexistence to alternative models and a Bayesian parametrization. Chapter two analyses via data simulation how an interaction chain between three species changes the coexistence outcome of a focal pair. Chapter three examines how the interaction between plants and pollinators is altered through pollination interference. The uncertainty surrounding species interactions might be intrinsic to how natural communities operate, however our theories and predictions should be robust enough to consider the noise that comes from our study of nature. 