% Publications - a page listing research articles written using content in the thesis

\pdfbookmark[1]{Preface}{Preface} % Bookmark name visible in a PDF viewer

\chapter*{Preface} % Publications page text
My thesis has been prepared as a collection of three standalone scientific articles. Each chapter is a standalone piece of research and, therefore, I only provide a general Introduction and Conclusion chapters linking the three chapters together. In  \autoref{Intro}, I focus on describing how my three chapters are connected. In \autoref{Conclusion}, I focus on summarising the results from each of my thesis chapters and their combined implications in both how we study interactions and their consequences for diversity maintenance. Finally, I further expand on new ideas beyond those presented in the different chapters to discuss about the future steps moving forward.


At the time of thesis submission, each of these three articles are at different stages of the publication process.

\autoref{Bee_foraging}: ``The context dependency of pollinator interference: how environmental conditions and co-foraging species impact floral visitaion'' was published in May 2021 in the journal \textit{Ecology Letters} in volume 24, no. 7, pages 1443--1454.

\autoref{Bayesian_competition}: ``The interplay of environmental conditions, parameter sensitivity and structural sensitivity in predictions of species coexistence'' is in preparation for submission to \textit{Ecology Letters}.

\autoref{Coexistence_alleles}: ``Quantifying the relative contributions of environmental fluctuations to the maintenance of a sexually antagonistic polymorphism'' is in preparation for submission to \textit{The American Naturalist}.


%\begin{refsection}[ownpubs]
%    \small
%    \nocite{*} % is local to to the enclosing refsection
%    \printbibliography[heading=none]
%\end{refsection}

%\emph{Attention}: This requires a separate run of \texttt{bibtex} for your \texttt{refsection}, \eg, \texttt{ClassicThesis1-blx} for this file. You might also use \texttt{biber} as the backend for \texttt{biblatex}. See also
